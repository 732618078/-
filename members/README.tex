\documentclass[]{article}
\usepackage{lmodern}
\usepackage{amssymb,amsmath}
\usepackage{ifxetex,ifluatex}
\usepackage{fixltx2e} % provides \textsubscript
\ifnum 0\ifxetex 1\fi\ifluatex 1\fi=0 % if pdftex
  \usepackage[T1]{fontenc}
  \usepackage[utf8]{inputenc}
\else % if luatex or xelatex
  \ifxetex
    \usepackage{mathspec}
  \else
    \usepackage{fontspec}
  \fi
  \defaultfontfeatures{Ligatures=TeX,Scale=MatchLowercase}
\fi
% use upquote if available, for straight quotes in verbatim environments
\IfFileExists{upquote.sty}{\usepackage{upquote}}{}
% use microtype if available
\IfFileExists{microtype.sty}{%
\usepackage{microtype}
\UseMicrotypeSet[protrusion]{basicmath} % disable protrusion for tt fonts
}{}
\usepackage[margin=1in]{geometry}
\usepackage{hyperref}
\hypersetup{unicode=true,
            pdftitle={openbiox members},
            pdfborder={0 0 0},
            breaklinks=true}
\urlstyle{same}  % don't use monospace font for urls
\usepackage{graphicx,grffile}
\makeatletter
\def\maxwidth{\ifdim\Gin@nat@width>\linewidth\linewidth\else\Gin@nat@width\fi}
\def\maxheight{\ifdim\Gin@nat@height>\textheight\textheight\else\Gin@nat@height\fi}
\makeatother
% Scale images if necessary, so that they will not overflow the page
% margins by default, and it is still possible to overwrite the defaults
% using explicit options in \includegraphics[width, height, ...]{}
\setkeys{Gin}{width=\maxwidth,height=\maxheight,keepaspectratio}
\IfFileExists{parskip.sty}{%
\usepackage{parskip}
}{% else
\setlength{\parindent}{0pt}
\setlength{\parskip}{6pt plus 2pt minus 1pt}
}
\setlength{\emergencystretch}{3em}  % prevent overfull lines
\providecommand{\tightlist}{%
  \setlength{\itemsep}{0pt}\setlength{\parskip}{0pt}}
\setcounter{secnumdepth}{0}
% Redefines (sub)paragraphs to behave more like sections
\ifx\paragraph\undefined\else
\let\oldparagraph\paragraph
\renewcommand{\paragraph}[1]{\oldparagraph{#1}\mbox{}}
\fi
\ifx\subparagraph\undefined\else
\let\oldsubparagraph\subparagraph
\renewcommand{\subparagraph}[1]{\oldsubparagraph{#1}\mbox{}}
\fi

%%% Use protect on footnotes to avoid problems with footnotes in titles
\let\rmarkdownfootnote\footnote%
\def\footnote{\protect\rmarkdownfootnote}

%%% Change title format to be more compact
\usepackage{titling}

% Create subtitle command for use in maketitle
\newcommand{\subtitle}[1]{
  \posttitle{
    \begin{center}\large#1\end{center}
    }
}

\setlength{\droptitle}{-2em}

  \title{openbiox members}
    \pretitle{\vspace{\droptitle}\centering\huge}
  \posttitle{\par}
    \author{}
    \preauthor{}\postauthor{}
    \date{}
    \predate{}\postdate{}
  

\begin{document}
\maketitle

\subsection{openbiox小组成员介绍(排名不分先后)}\label{openbiox}

\subsubsection{李剑峰}

openbiox小组发起人,生信技能树社区核心成员,是开源生物信息学和开放科学思想的坚定支持者、传播者和践行者。上海交通大学医学院附属瑞金医院,医学基因组学国家重点实验室,上海血液学研究所博士生,主要研究方向是生物信息学和白血病。\\
\href{https://life2cloud.com}{个人博客} 和
\href{https://www.zhihu.com/people/life2cloud}{知乎主页}

\subsubsection{曾健明}

在
\href{http://www.bio-info-trainee.com/}{生信菜鸟团博客}坚持5年持续分享生物信息学相关领域技术知识;
创办广为人知的
\href{http://www.biotrainee.com/}{生信技能树论坛},是目前全球最大的华语生物信息学专业论坛及公众号;
已经培养了近1万名生信工程师,完成
\href{https://space.bilibili.com/338686099}{全套生物信息学入门及进阶视频教程};
目标
\href{https://mp.weixin.qq.com/s/E9ykuIbc-2Ja9HOY0bn_6g}{培养10万名生信工程师},
\href{https://static.dingtalk.com/media/lALPAuoR5z1wDQ3NA_DNAoA_640_1008.png}{2019全国巡讲至少到达60个城市}。

\subsubsection{赵龙飞}

郑州大学生物信息学博士

\subsubsection{许志骁}

华东理工大学在读博士。目前主要进行全基因组和比较基因组方面的分析。负责了实验室的服务器平台的搭建,构建了本课题组的数据分析流程。后期对实验室成员进行了生信相关技能的培训,发展了本课题组的生信团队。

\subsubsection{张宇}

本科期间主要研究方向为黄酮醇合成酶的功能验证;现为中国农业科学院硕士研究生,研究方向为代谢组与转录组联用的抗逆代谢调节物检测与验证,请见
\href{https://orcid.org/0000-0001-8506-5222}{个人ORCID}

\subsubsection{乔士达}

上海药物研究所 药学硕士

\subsubsection{王舒}

复旦大学华山医院血液科博士

\subsubsection{张佩}

河北大学与国家蛋白质科学中心联合培养硕士研究生,在国家蛋白质科学中心学习3年。热爱科研,热爱写代码的科研码农。有理想有情怀有热情的,偶尔有些小文艺。

\subsubsection{王琪}

R语言办公,C语言生产

\subsubsection{魏晨}

大连医科大学 生物医学工程 本科

\subsubsection{王灵}

本科临床,硕士流统,目前基础,未来不知

\subsubsection{包日强}

上海交通大学医学院附属瑞金医院内分泌在读博士

\subsubsection{贾敏}

深圳瑞奥康晨科技有限公司 生信分析工程师

\subsubsection{丁雨}

哈尔滨医科大学 生物医学工程 硕士

\subsubsection{王海濤}

National Cancer Centre Singapore 新加坡國立癌症中心

\subsubsection{陈颖珊}

中国科学院在读博士

\subsubsection{袁也}

云南农业大学 植物病理学硕士

\subsubsection{阚科佳}

德国海德堡大学曼海姆医学院血管外科博士

\subsubsection{杨芮}

山西医科大学第二临床医学院外科学硕士研究生二年级

\subsubsection{梁其云}

山东大学海洋学院微生物学在读博士(一年级)

\subsubsection{韩漾}

中国科学院大学 生物信息学博士在读

\subsubsection{郑东旭}

广州医科大学附属第五医院

\subsubsection{王淑轩}

电子科技大学生物物理学硕士

\subsubsection{王书喆}

苏黎世联邦理工计算化学博士在读

\subsubsection{赵清波}

上海交通大学在读博士,研究方向统计基因组学与生物信息学,专注于遗传多样性保护理论与方法、选择信号、GWAS/GS算法模型开发与应用。热爱统计与编程,喜欢文学,乐于分享与传播。

\subsubsection{曾鑫}

东京大学计算生物学与医学硕士生在读

\subsubsection{王诗翔}

上海科技大学博士研究生,研究和开发癌症生物标志物

\subsubsection{王义冠}

澳大利亚昆士兰大学生物系在读博士

\subsubsection{代雨婷}

上海交通大学生物学博士

\subsubsection{曾海銮}

复旦大学附属中山医院内分泌科,复旦大学人类表型组研究院,博士生

\subsubsection{姜思彤}

东北农业大学作物遗传育种硕士

\subsubsection{郭奕鑫}

浙江大学医学院免疫所硕士

\subsubsection{余超然}

上海交通大学瑞金医院胃肠外科在读博士

\subsubsection{熊逸}

中南大学临床医学博士

\subsubsection{张子颖}

荷兰瓦赫宁恩大学生物信息学硕士

\subsubsection{张楠}

哈尔滨医科大学生物信息学硕士

\subsubsection{王慧美}

复旦大学神经生物学硕士


\end{document}
