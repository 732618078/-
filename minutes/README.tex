\documentclass[]{article}
\usepackage{lmodern}
\usepackage{amssymb,amsmath}
\usepackage{ifxetex,ifluatex}
\usepackage{fixltx2e} % provides \textsubscript
\ifnum 0\ifxetex 1\fi\ifluatex 1\fi=0 % if pdftex
  \usepackage[T1]{fontenc}
  \usepackage[utf8]{inputenc}
\else % if luatex or xelatex
  \ifxetex
    \usepackage{mathspec}
  \else
    \usepackage{fontspec}
  \fi
  \defaultfontfeatures{Ligatures=TeX,Scale=MatchLowercase}
\fi
% use upquote if available, for straight quotes in verbatim environments
\IfFileExists{upquote.sty}{\usepackage{upquote}}{}
% use microtype if available
\IfFileExists{microtype.sty}{%
\usepackage{microtype}
\UseMicrotypeSet[protrusion]{basicmath} % disable protrusion for tt fonts
}{}
\usepackage[margin=1in]{geometry}
\usepackage{hyperref}
\hypersetup{unicode=true,
            pdftitle={minutes},
            pdfborder={0 0 0},
            breaklinks=true}
\urlstyle{same}  % don't use monospace font for urls
\usepackage{graphicx,grffile}
\makeatletter
\def\maxwidth{\ifdim\Gin@nat@width>\linewidth\linewidth\else\Gin@nat@width\fi}
\def\maxheight{\ifdim\Gin@nat@height>\textheight\textheight\else\Gin@nat@height\fi}
\makeatother
% Scale images if necessary, so that they will not overflow the page
% margins by default, and it is still possible to overwrite the defaults
% using explicit options in \includegraphics[width, height, ...]{}
\setkeys{Gin}{width=\maxwidth,height=\maxheight,keepaspectratio}
\IfFileExists{parskip.sty}{%
\usepackage{parskip}
}{% else
\setlength{\parindent}{0pt}
\setlength{\parskip}{6pt plus 2pt minus 1pt}
}
\setlength{\emergencystretch}{3em}  % prevent overfull lines
\providecommand{\tightlist}{%
  \setlength{\itemsep}{0pt}\setlength{\parskip}{0pt}}
\setcounter{secnumdepth}{0}
% Redefines (sub)paragraphs to behave more like sections
\ifx\paragraph\undefined\else
\let\oldparagraph\paragraph
\renewcommand{\paragraph}[1]{\oldparagraph{#1}\mbox{}}
\fi
\ifx\subparagraph\undefined\else
\let\oldsubparagraph\subparagraph
\renewcommand{\subparagraph}[1]{\oldsubparagraph{#1}\mbox{}}
\fi

%%% Use protect on footnotes to avoid problems with footnotes in titles
\let\rmarkdownfootnote\footnote%
\def\footnote{\protect\rmarkdownfootnote}

%%% Change title format to be more compact
\usepackage{titling}

% Create subtitle command for use in maketitle
\providecommand{\subtitle}[1]{
  \posttitle{
    \begin{center}\large#1\end{center}
    }
}

\setlength{\droptitle}{-2em}

  \title{minutes}
    \pretitle{\vspace{\droptitle}\centering\huge}
  \posttitle{\par}
    \author{}
    \preauthor{}\postauthor{}
    \date{}
    \predate{}\postdate{}
  

\begin{document}
\maketitle

\section{openbiox小组例会纪要}\label{openbiox}

\subsection{2019.03.08 20:30 线上会议}

\begin{enumerate}
\def\labelenumi{\arabic{enumi}.}
\tightlist
\item
  本次例会全体常委暂时没有其他需要推荐加入openbiox的人选。
\item
  新收到的申请邮件中,*楠通过半数常委同意,予以纳入。
\item
  补充圆形Logo,小组公约格式调整,拟明天下午发布openbiox小组公约1.0。
\item
  实行轮值。常务委员两人一组,轮值两周内的平时活动、收集议题、整理创意和一次例会的组织与记录。建议自由选择轮值伙伴,全部轮完开启下一轮。
\item
  Openbiox小组任何成员都可以提出创意,是否公开由创意提出者决定。创意池是否建设待定。
\item
  本次轮值:**峰,**銮。下次轮值:**婷,**浩。
\end{enumerate}

\subsection{2019.03.23 20:30 线上会议}\label{-1}

\begin{enumerate}
\def\labelenumi{\arabic{enumi}.}
\tightlist
\item
  本次例会全体常委暂时没有其他需要推荐加入openbiox的人选。
\item
  新收到的申请邮件中,*欣,*恺,**楠三名成员通过半数常委同意,予以纳入。
\item
  制定项目小组定期汇报制度,两周一次与常委例会交替进行。
\item
  成员参与活动情况考察拟采用个人月报与项目负责人汇报结合的形式。
\item
  纳新机制暂时保持现有的机制不变,等项目运行起来之后再依据实际情况而定。
\item
  制定了项目组实践过程中遇到的问题反馈机制,采用项目内自行组织讨论与Openbiox小组内集思广益结合的形式。问题解决后,可总结经验输出内容发布到微信公众号与知乎专栏。
\item
  提出每周一问的活动,每个人一周在群里发布一个问题,可以是任意主题,促进思考和交流.
\item
  讨论了小项目发起方式,小组成员可以发起一些参与度高,比较有趣味性的小项目。
\item
  制定了轮值委员会实习制度,推荐优秀的组员进入轮值委员会实习,参与常委会议的讨论与投票。
\item
  讨论了通用技能培训开展的形式,包括:主题学习周、Openbiox小组内每周一问、专题学习资料Wiki归档。
\item
  经过常委讨论,建议每个成员最多参与 2-3 个项目。
\item
  本次轮值:**婷,**浩。下次轮值:*朋,**洋。
\end{enumerate}

\subsection{2019.04.06 20:30 线上会议}\label{-2}

\subsubsection{议题:}

\begin{enumerate}
\def\labelenumi{\arabic{enumi}.}
\tightlist
\item
  项目进展如何汇报以及监管
\item
  4月下旬线下活动组织形式
\item
  如何统计成员参与情况
\item
  是否需要再wiki仓库公布资金捐赠渠道,经费支出流程如何优化
\item
  主题学习周以及每周一问如何开展
\end{enumerate}

\subsubsection{讨论结果:}

\begin{enumerate}
\def\labelenumi{\arabic{enumi}.}
\tightlist
\item
  每次例会后项目进展由负责人按照模板直接发送至全员群,项目组成员、全部成员均可以发表意见并参与监督
\item
  有意者请加入剑锋的微信群交流讨论
\item
  主要从两方面统计成员参与情况:1)钉钉数据库中成员参与情况:包括域名邮箱,GitHub账号设置,个人签名以及资源捐赠,每周任务完成度等
  2)具体项目参与情况:由项目负责人统计,主要平台为GitHub,评价频率根据项目进度,在一月至两月之间浮动
\item
  捐赠渠道可以在GitHub中wiki仓库查看。支出流程现优化如下:申请人提出申请,当日轮值委员负责在常委中发起投票表决,通过后由经费卡支出。
\item
  ``主题学习周''以及
  ``每周一问''均以任务形式发布在全员群,频率分别为两周一次和一周一次。常委会负责对收集的主题学习内容进行投票表决,通过的主题由提出人负责在全员群中发起,形式包括但不限于:学习资源分享,答疑,参与比赛,头脑风暴以及组建新项目等
\end{enumerate}

本次轮值:**朋,**洋。下次轮值:*泽,**成。


\end{document}
